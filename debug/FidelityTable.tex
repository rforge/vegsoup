% latex.default(tex, file = filename, caption = caption, rowname = NULL,      booktabs = TRUE, longtable = TRUE, lines.page = nrow(tex),      here = TRUE, col.just = col.just) 
%

\setlongtables


\begin{longtable}{p{70mm}p{10mm}p{10mm}p{10mm}} \caption{Fidelity table for 2 partitions.  Relevees per partition:  1:6, 2:6}\tabularnewline
 \toprule
\multicolumn{1}{c}{Taxon}&\multicolumn{1}{c}{Layer}&\multicolumn{1}{c}{1}&\multicolumn{1}{c}{2}\tabularnewline
\midrule
\endfirsthead
\caption[]{\em (continued)} \tabularnewline
\midrule
\multicolumn{1}{c}{Taxon}&\multicolumn{1}{c}{Layer}&\multicolumn{1}{c}{1}&\multicolumn{1}{c}{2}\tabularnewline
\midrule
\endhead
\midrule
\endfoot
\label{tex}
Carex alba&hl&83*&0*\tabularnewline
Viola hirta&hl&83*&0*\tabularnewline
Anthericum ramosum&hl&100&100\tabularnewline
Carex ornithopoda subsp. ornithopoda&hl&100&100\tabularnewline
Juniperus sabina&hl&100&100\tabularnewline
Laserpitium latifolium subsp. latifolium&hl&100&100\tabularnewline
Teucrium montanum&hl&100&100\tabularnewline
Calamagrostis varia&hl&83&100\tabularnewline
Globularia cordifolia s.str.&hl&83&100\tabularnewline
Sesleria caerulea s.str.&hl&100&83\tabularnewline
Vincetoxicum hirundinaria&hl&83&100\tabularnewline
Amelanchier ovalis&hl&67&100\tabularnewline
Buphthalmum salicifolium&hl&83&83\tabularnewline
Carduus defloratus subsp. viridis&hl&100&67\tabularnewline
Clinopodium alpinum subsp. alpinum&hl&100&67\tabularnewline
Galium anisophyllon&hl&83&83\tabularnewline
Polygala chamaebuxus&hl&83&83\tabularnewline
Polygonatum odoratum&hl&83&83\tabularnewline
Thymus praecox subsp. polytrichus&hl&67&100\tabularnewline
Allium lusitanicum&hl&83&67\tabularnewline
Gypsophila repens&hl&67&83\tabularnewline
Juniperus communis subsp. communis&hl&67&83\tabularnewline
Sedum album&hl&83&67\tabularnewline
Erysimum sylvestre s.str.&hl&83&50\tabularnewline
Helianthemum nummularium subsp. obscurum&hl&67&50\tabularnewline
Hippocrepis comosa&hl&50&67\tabularnewline
Scabiosa columbaria s.str.&hl&33&83\tabularnewline
Silene nutans subsp. nutans&hl&50&67\tabularnewline
Squamarina lamarckii&hl&50&67\tabularnewline
Asplenium ruta-muraria&hl&33&67\tabularnewline
Aster alpinus&hl&50&50\tabularnewline
Berberis vulgaris&hl&33&50\tabularnewline
Hieracium bifidum&hl&33&50\tabularnewline
Poa molinieri&hl&50&33\tabularnewline
Rhamnus cathartica&hl&33&50\tabularnewline
Rosa micrantha&hl&67&17\tabularnewline
Leontodon hispidus subsp. pseudocrispus&hl&33&33\tabularnewline
Origanum vulgare subsp. vulgare&hl&50&17\tabularnewline
Picea abies&hl&67&.\tabularnewline
Potentilla caulescens subsp. caulescens&hl&17&50\tabularnewline
Saxifraga paniculata&hl&50&17\tabularnewline
Sorbus aria agg.&hl&67&.\tabularnewline
Mercurialis perennis s.str.&hl&50&.\tabularnewline
Rosa tomentosa&hl&33&17\tabularnewline
Thesium alpinum&hl&33&17\tabularnewline
Echium vulgare subsp. vulgare&hl&33&.\tabularnewline
Galium mollugo s.str.&hl&17&17\tabularnewline
Larix decidua&hl&33&.\tabularnewline
Rosa subcollina&hl&33&.\tabularnewline
Asplenium trichomanes s.lat.&hl&17&.\tabularnewline
Campanula trachelium&hl&17&.\tabularnewline
Carex digitata&hl&.&17\tabularnewline
Cotoneaster tomentosus&hl&17&.\tabularnewline
Crepis alpestris&hl&17&.\tabularnewline
Erica carnea&hl&17&.\tabularnewline
Fraxinus excelsior&hl&17&.\tabularnewline
Leontodon incanus&hl&17&.\tabularnewline
Melica nutans s.str.&hl&17&.\tabularnewline
Rosa canina&hl&17&.\tabularnewline
Valeriana tripteris&hl&17&.\tabularnewline
Juniperus communis subsp. communis&wl&100&50\tabularnewline
Amelanchier ovalis&wl&67&67\tabularnewline
Picea abies&wl&50&17\tabularnewline
Rhamnus cathartica&wl&.&67\tabularnewline
Larix decidua&wl&50&.\tabularnewline
Sorbus aria agg.&wl&.&50\tabularnewline
Fraxinus excelsior&wl&33&.\tabularnewline
Rosa canina&wl&17&17\tabularnewline
Rosa glauca&wl&17&17\tabularnewline
Rosa micrantha&wl&33&.\tabularnewline
Berberis vulgaris&wl&.&17\tabularnewline
Corylus avellana&wl&.&17\tabularnewline
Cotoneaster tomentosus&wl&17&.\tabularnewline
Populus tremula&wl&17&.\tabularnewline
Rosa subcollina&wl&17&.\tabularnewline
\bottomrule
 \end{longtable}
  
