\documentclass[9pt,BCOR=0mm,a4paper,oneside,DIV=11]{scrartcl}
\usepackage[utf8]{inputenc}
\usepackage[ngerman]{babel}
\usepackage[ngerman]{varioref}
\usepackage{scrpage2}
\usepackage{booktabs}
\usepackage{longtable}
\usepackage{multicol}

\usepackage{lscape}
\usepackage{rotating}

\renewcommand{\floatpagefraction}{0.75}

\usepackage[scaled]{helvet}
\renewcommand*\familydefault{\sfdefault}

\title{Vegetationstabellen zum Fachbericht Pflanzen und Lebensräume}
\subject{UVE Ferienpark Resort Gastein}
\author{Thomas Eberl \& Roland Kaiser}
\date{\today}

\begin{document}
\maketitle
%\begin{landscape}

\noindent
Die Vegetationstabelle mit ergänzenden Angaben zur Aufnahmefläche ist in Form von zwei korrespondierenden Tabellen dargestellt. Die Aufnahmenummer stellt den Bezug zwischen den Tabellen her. Die erste Tabelle beinhaltet Angaben zur Struktur der Pflanzenbestände (Deckung und Höhe der Vegetationsschichten in Prozent bzw. Metern, Aufnahmefläche in Quadratmetern, Exposition und Inklination).
Der Inhalt der zweiten Tabelle gibt die Deckungswerte der Pflanzenarten, geordnet nach Bestandesschichten, wider. Die Zuordnung der Vegetationsaufnahmen zu Pflanzengesellschaften ist der textlichen Ausführung des Fachberichts zu entnehmen.

\pagestyle{useheadings}
\input{LaTex_input.tex}
	
%\end{landscape}
\end{document}
